
\clearpage{}
\section{Semantics}
The key aspects of our system are captured in the rule structure and authority mechanism. For expositional purposes the term language has been specified as a version of simply typed lambda calculus. For a concrete implementation it may be preferrable to define rule bodies in terms of a well defined bytecode such as Web Assembly. \TODO{Cite Corda uses a determinised JVM to specify its contracts}

\AMOS{
In the execution stage, why is @say@ a side-effect that updates the facts list (EvSay), rather than just having the rule term evaluate to a list of factoids?
I think in a coloured petri net the transition function is a pure function from input values to output values, so I'd expect here a pure function from matched input factoids to output factoids.
}

% ---------------------------------------------------------
\begin{figure}
\begin{small}
\begin{alltt}
Name   ::= ...
Ctor   ::= ...
Label  ::= ...

Decl   ::= rule Name Rule
Rule   ::= await Guard+ to Action
Guard  ::= Ctor as Var where Term

Type   ::= unit | text | symbol | nat | party
        |  set Type | Type \(\to\) Type

Term   ::= Var | Term Term | \(\lambda\) Var : Type \(\to\) Term
        |  [ (Label = Term)* ] | Term . Label

Value  ::= Lit | \(\lambda\) Var : Type \(\to\) Term
        |  [ (Label = Value)* ]

Action ::= done | Action ; Action
        |  say Ctor [ (Label = Exp)* ]
           by  Term  use Term  for Term  obs Term
\end{alltt}
\end{small}

\caption{Core Language Grammar}
\end{figure}


% -----------------------------------------------------------------------------
% Dynamic Semantics
% -----------------------------------------------------------------------------


\begin{figure}
$$
\begin{array}{cl}
\fbox{$\jOk{Env}{Rule}$}
\\[2ex]


\ruleI  {    \jTypeMatch{\cdot}{\Sigma}{H^n}{\Gamma}
         \qq \jType{\Gamma}{M}{\kUnit} }
        {\jOk{\Sigma}{\krule~ N~ \kawait~ H^n~ \kto~ M}}
\end{array}
$$

$$
\begin{array}{cl}
\jTypeMatch{\Gamma}{\Sigma}{\cdot}{\Gamma}
\\[1ex]
\ruleI  {   \jTypeMatch{\Gamma}{\Sigma}{H}{\Gamma'}
        \qq \jTypeMatch{\Gamma'}{\Sigma}{H^n}{\Gamma''} }
        {\jTypeMatch{\Gamma}{\Sigma}{H~H^n}{\Gamma''}}
\end{array}
$$

$$
\begin{array}{cl}
\ruleI  {\begin{array}{ll}
            \jTypeGather{\Gamma}{\Sigma}{X}{G}{\Gamma'}
        \\  \jOk{\Gamma'}{T} \quad \jOk{\Gamma'}{C} \quad \jOk{\Gamma'}{I}
         \end{array}
        }
        {\jTypeMatch{\Gamma}{\Sigma}{X~ \kfrom~ G~ T~ C~ I}{\Gamma'}}
\end{array}
$$

$$
\begin{array}{cl}
\ruleI  {    (N : \tau) \in \Sigma
         \quad \Gamma' = \Gamma,~ X : \kFact~ \tau
         \quad \{ \jType{\Gamma'}{M_i}{\kBool} \}^{i \gets 1 .. n}
        }
        {\jTypeGather{\Gamma}{\Sigma}{X}{\kwhen~N~M^n}{\Gamma'}}
\end{array}
$$

$$
\begin{array}{ccc}
% -- Select
        \jOk    {\Gamma}{\kany}
&       \jOk    {\Gamma}{\kretain}
&       \jOk    {\Gamma}{\ksame}
\\[2ex]
        \ruleI  {\jType{\Gamma}{M}{\kNat}}
                {\jOk  {\Gamma}{\kfirst~ M}}
&       \ruleI  {\jType{\Gamma}{M}{\kNat}}
                {\jOk{\Gamma}{\kconsume~ M}}
&       \ruleI  {\jType{\Gamma}{M}{\trm{\kAuth}}}
                {\jOk{\Gamma}{\kgain~ M}}
\end{array}
$$

$$
\begin{array}{cl}
\fbox{$\jTypeAction{Env}{Env}{Term}{Type}$}
\\[2ex]

\jTypeAction{\Gamma}{\Sigma}{\kdone}{\kUnit}

\\[2ex]
\ruleI  {   \jTypeAction{\Gamma}{\Sigma}{M}{\kUnit}
        \qq \jTypeAction{\Gamma}{\Sigma}{M'}{\tau} }
        { \jTypeAction{\Gamma}{\Sigma}{M ~;~ M'}{\tau} }

\\[2ex]
\ruleI  { \begin{array}{ll}
           (N : \tau) \in \Sigma
           \\ \jType{\Gamma}{M_{by}} {\kSet~ \kParty}
            & \jType{\Gamma}{M_{obs}}{\kSet~ \kParty}
           \\ \jType{\Gamma}{M_{use}}{\kSet~ \kSymbol}
            & \jType{\Gamma}{M_{num}}{\kNat~}
          \end{array}
        }
        {\begin{array}{ll}
         \Gamma ~|~ \Sigma ~\vdash
                   & \hspace{-1ex} \ksay~ N_{tag}~ M_{payload}
                \\ & \hspace{-1ex} \kby~  M_{by}  ~~\kobs~ M_{obs} \kuse~ M_{use}~ \knum~ M_{num}
                :: \kUnit
         \end{array}
        }
\end{array}
$$

$$
\begin{array}{cl}
\fbox{$\jType{Env}{Term}{Type}$}
\\[2ex]
\ruleI  { (X : \tau) \in \Gamma }
        {\jType{\Gamma}{X}{\tau}}
\qq
\ruleI  {   \jType{\Gamma}{M}{\tau \to \tau'}
        \qq \jType{\Gamma}{M'}{\tau} }
        { \jType{\Gamma}{M~ M'}{\tau'} }
\qq \dots
\end{array}
$$

\caption{Static Semantics}
\label{f:Statics}
\end{figure}


% -----------------------------------------------------------------------------
% Dynamic Semantics
% -----------------------------------------------------------------------------

\begin{figure*}
$$
\begin{array}{cl}

% -- Fire -----------------------------
\fbox{$\jFire{Auth}{Store}{Rule}{Factoids}{Factoids}{Store}$}
\\[2ex]
\ruleI
{ \begin{array}{ll}
  &  \bigwedge \{ A_{has} \supseteq \trm{auth-by}~D~|~D \in D_{new}^m \}
  \\[1ex] \jMatches{N}{A_{sub}}{S}{\cdot}{H^n}{A_{has}}{D_{spent}^n}{E'}
  &  ~~~ \jExec{E'}{M}{\kunit}{D_{new}^m}
  \end{array}
}
{\jFire {A_{sub}}{S}
        {\krule~N~\kawait~H^n~\kto~M}
        {D_{spent}^n}{D_{new}^m}{(S \setminus D^n_{spent}) \uplus D_{new}^m}
}
& (\trm{EvFire})
\\[4ex]


% -- matches --------------------------
\fbox{$\jMatches{Name}{Auth}{Store}{Env}{Matches}{Auth}{Factoids}{Env}$}
\\[2ex]

\jMatches{N}{A}{S}{E}{\cdot}{\emptyset}{\emptyset}{E}
& (\trm{EvMatchNil})
\\[2ex]

\ruleI
{ \begin{array}{ll}
       \jMatch{N_{rule}}{A_{sub}}{S~}{E~}{H~~}{A_{gain}}{D_{spent}}{S'}{E'}
  \\[0.5ex]
       \jMatches{N_{rule}}{A_{sub}}{S'}{E'}{H^n}{A_{gain}'}{D_{spent}'}{E''~~~~~  }
  \end{array}
}
{ \jMatches{N_{rule}}{A_{sub}}{S}{E}{H ~ H^n}
           {A_{gain}  \cup A_{gain}'}
           {D_{spent} \cup D_{spent}'}
           {E''}
}
& (\trm{EvMatchCons})
\\[4ex]


% -- match ----------------------------
\fbox{$\jMatch{Name}{Auth}{Store}{Env}{Match}{Auth}{Factoid}{Store}{Env}$}
\\[2ex]

% -- MatchGuard
\ruleI
{ \begin{array}{ll}
     \jGather{A_{sub}}{S}{E}{X}{G}{F^n}
  &  ~~ \jGain{F}{E'}{I}{A_{gain}}
  \\ ~~~~~~~~~\jSelect{F^n}{E}{X}{T}{F~~}
  &  ~~ E' = E, X \mapsto F
  \\ ~~~~ \jConsume{F}{S}{E'}{C}{W}{S'}
  &  ~~ N_{rule} \in \trm{rules}~ F
  \end{array}
}
{   \jMatch{N_{rule}}{A_{sub}}{S}{E}{X~\kfrom~G~T~C~I}{A_{gain}}{(F,W)}{S'}{E'}
}
& (\trm{EvMatch})
\\[4ex]


% -- gather ----------------------------
\fbox{$\jGather{Auth}{Store}{Env}{Var}{Gather}{Facts}$}
\\[2ex]

\ruleI
{ F^n = \left \{
  \begin{array}{rll}
        F & |~~ (F \mapsto W) \in S
     \\    & ,~~ \trm{name}~ F = N_{fact}
             ,~~ \trm{sees}~ A_{sub}~ F
             ,~~  W \geq 1
     \\    & ,~~ (\jEval{E,~X \mapsto F}{M^m}{\ktrue})
     \end{array}
  \right \}
}
{   \jGather{A_{sub}}{S}{E}{X}{\kwhen~N_{fact}~M^m}{F^n}
}
& (\trm{EvWhen})
\\[4ex]


%% -- select ---------------------------
\fbox{$\jSelect{Facts}{Env}{Var}{Select}{Fact}$}
\\[2ex]

\ruleI
{ F \in F^n }
{ \jSelect{F^n}{E}{X}{\kany}{F} }
\qq
\ruleI
{ \begin{array}{ll}
      (V', F')  \in D^m
  \qq V' = \trm{minimum}~ \{ V ~|~ (V, \_) \in D^m \}
  \\  ~~~~~~~ D^m \in \{ (V, F) ~|~ F \in F^n, (\jEval{E, X \mapsto F}{M}{V} ) \}
  \end{array}
}
{ \jSelect{F^n}{E}{X}{\kfirst~ M}{F} }
& (\trm{EvAny/First})
\\[4ex]


%% -- consume --------------------------
\fbox{$\jConsume{Fact}{Store}{Env}{Consume}{Weight}{Store}$}
\\[2ex]

 \jConsume{F}{S}{E}{\kretain}{0}{S}
 & (\trm{EvRetain})
\\[2ex]

\ruleI
{     \jEval{E}{M}{W_{need}}
  \qq W_{avail} \geq W_{need}
}
{ \jConsume
        {F}{S[F \mapsto W_{avail}]}{E}
        {\kconsume~ M}
        {W_{need}}{S[F \mapsto W_{avail} - W_{need}]}
}
& (\trm{EvConsume})
\\[4ex]


%% -- gain ----------------------------
\fbox{$\jGain{Fact}{Env}{Gain}{Auth}$}
\\[2ex]
\jGain{F}{E}{\ksame}{\emptyset}
\qq
\ruleI
{     \jEval{E}{M}{A}
  \qq A \subseteq \trm{auth-by}~ F
}
{ \jGain{F}{E}{\kgain~ M}{A}
}
& (\trm{EvSame/Gain})

\end{array}
$$

\caption{Dynamic Semantics}
\label{f:Dynamics}
\end{figure*}




\begin{figure}
\begin{small}
\begin{alltt}
fact  Coin   [stamp: Symbol, holder: Party]
fact  Offer  [id: Symbol, giver: Party, receiver: Party]
fact  Accept [id: Symbol, accepter: Party]

node  Bank

rule  transfer
await Accept  as accept
 and  Offer   as offer
      where   accept.id       == offer.id,
              accept.accepter == offer.receiver
 and  Coin    as coin
      where   coin.holder     == offer.giver
to
      say Coin [ stamp  = coin.stamp
               , holder = offer.receiver]
      by  [auth| Bank, offer.receiver]
      use (elem offer.receiver)

scenario Bank, Alice, Bob
to do say Coin   [ stamp = 'Coin1001, holder = Alice]
      by  [auth| Bank, Alice]
      use (elem Alice)

      say Offer  [ id    = '1234
                 , giver = Alice, receiver = Bob]
      by  Alice use (elem Bob)

      say Accept [ id = '1234,    accepter = Bob]
      by  Bob   use (const true)
\end{alltt}
\end{small}

\caption{Coin Transfer Workflow}
\label{f:CoinTransferDesugared}
\end{figure}

\subsection{Lemmas}

\begin{lemma}
New facts are included in the new store (trivial):
$$
\jFire{A_{sub}}{S}{R}{D_{spent}^n}{D_{new}^m}{S'}
\implies
D_{new}^m \subseteq S'
$$
\end{lemma}
(Using the $\subseteq$ notation to refer to the multiset subset operation here -- would it be better to unroll it as something like $\forall (x,w) \in D_{new}^m.\ (x,w') \in S' \wedge w \le w'$ ?)

\begin{lemma}
Rules only spend what is in the store (pretty trivial):
$$
\jFire{A_{sub}}{S}{R}{D_{spent}^n}{D_{new}^m}{S'}
\implies
D_{spent}^n \subseteq S
$$
\end{lemma}

\begin{theorem}
Frame constriction: if a rule succeeds, the rule would succeed, resulting in the same value, in a store containing only what it spent:
$$
\begin{array}{c}
\jFire{A_{sub}}{S}{R}{D_{spent}^n}{D_{new}^m}{S'}
\implies \\
\jFire{A_{sub}}{D_{spent}^n}{R}{D_{spent}^n}{D_{new}^m}{S'}
\end{array}
$$
\end{theorem}
This theorem statement isn't quite correct at the moment, because it's ignoring retained/non-consumed values. I will update this rule to find the non-consumed values in the spent facts and copy them from the original store after I have updated the formalisation. I think updating the formalisation will expose some other lemmas.

\AMOS{will fill in other lemmas soon}


