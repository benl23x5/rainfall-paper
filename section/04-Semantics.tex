

% -----------------------------------------------------------------------------
\section{Semantics}
The Rainfall semantics is defined in terms of a core language where pattern matching is expressed as set comprehension style generators. For expositional purposes we express rule bodies using a version of Simply Typed Lambda Calculus (STLC) with records and sets, but a production implementation could also use a more expressive language like System-F, or a well defined byte code. The key ideas of our system are embodied in the rule structure and authority mechanism, while the language of rule bodies is arbitrary.


% -----------------------------------------------------------------------------
\begin{figure}
\begin{tabbing}
Mx \= MMMMx \= Mx \= MMM \= MMMM \kill
\\  $N,$ \> $\iName$    \> ::= \> ...
\\  $L,$ \> $\iLabel$   \> ::= \> ...
\\  $X,$ \> $\iVar$     \> ::= \> ...

\\[1ex]
    $R,$ \> $\iRule$    \> ::= \> $\krule~ \iName~ \kawait~ \iMatch+~ \kto~ \iTerm$
\\[0.5ex]
    $H,$ \> $\iMatch$   \> ::= \> $\iVar~ \kfrom~ \iName~ \kwhere~ \iTerm$
\\      \>              \>     \> \hspace{1.6em}
                                  $\kselect~ \iSelect~ \kconsume~ \iConsume~ \kgain~ \iGain$
\\[0.5ex]
   $C,$ \> $\iSelect$   \> ::= \> $\kany~ |~ \kfirst~ \iTerm~ ~|~ \klast~ \iTerm$
\\ $U,$ \> $\iConsume$  \> ::= \> $\knone~ |~ \iTerm$
\\ $I,$ \> $\iGain$     \> ::= \> $\knone~ |~ \iTerm$

\\[1ex]
   $T,$ \> $\iType$
        \> ::= \> $\kUnit~ |~ \kBool~ |~ \kNat~ |~ \kText~ |~ \kSymbol~ |~ \kParty$
\\ \>   \> ~|  \> $\kSet~ \iType~ |~ \kSets~ \iType~ |~ \kFact~ \iType~ |~ \kFACT$
\\ \>   \> ~|  \> $[ (\iLabel : \iType)* ] ~|~ \iType \to \iType~$

\\[1ex]
  $M,$  \> $\iTerm$
        \> ::=  \> $\iLiteral~ |~ \iVar~ |~ \iTerm~ \iTerm~ |~ \lambda \iVar : \iType.~ \iTerm$
\\ \>   \> ~|   \> $[ (\iLabel = \iTerm)* ]~ |~ \iTerm~ .~ \iLabel ~|~ \{|~ \iTerm* |\}$

\\[0.5ex] \>    \> ~|  \> $\ksay~ \iName~ \iTerm~$
\\        \>    \>     \> \hspace{0.5em} $\kby~ \iTerm~ \kobs~ \iTerm~ \kuse~ \iTerm~ \knum~ \iTerm$

\\[1ex]
  $V,$  \> $\iValue$
        \> ::=  \> $\iLiteral~ |~ \lambda \iVar : \iType.~ \iTerm$
\\      \>      \> ~|  \> $[ (\iLabel = \iValue)* ] ~|~ \{ \iValue* \}$
\\      \>      \> ~|  \> $\iFact ~|~ \{ \iFact \mapsto \iWeight \}$

\\[1ex]
  $L,$  \> $\iLiteral$
        \> ::= \> $\kunit$ ~|~ $\iBool$ ~|~ $\iNat$ ~|~ $\iText$ ~|~ $\iSymbol$ ~|~ $\iParty$
\end{tabbing}

\medskip
\begin{flushleft}
(primitive operators)
\end{flushleft}
\begin{tabbing}
MMMMMMMM \= Mx \= MMMMM \kill
   @fact'payload@$_T$ \> $:: \kFact~ T \to T$
\\ @fact'by@$_T$      \> $:: \kFact~ T \to \kSet~ \kParty$
\\ @fact'obs@$_T$     \> $:: \kFact~ T \to \kSet~ \kParty$
\\ @fact'use@$_T$     \> $:: \kFact~ T \to \kSet~ \kSymbol$
\\ @sets'union@$_T$    \> $:: \kSets~ T \to \kSets~ T \to \kSets~ T$
\end{tabbing}

\medskip
\begin{flushleft}
(environments)
\begin{tabbing}
Mx              \= MMMMMx       \= Mx \= MMMMMMM \kill
   $\Gamma,$    \> $\iEnv$     \> ::= \> $\cdot ~|~ \iEnv,~ \iVar : \iType$
\\ $\Sigma,$    \> $\iDecls$   \> ::= \> $\cdot ~|~ \iDecls,~ \iName : [(\iLabel : \iType)*]$
\end{tabbing}
\end{flushleft}

\caption{Core Language Grammar}
\label{f:Grammar}
\end{figure}


% -----------------------------------------------------------------------------
\begin{figure}
\begin{small}
\begin{alltt}
rule  transfer
await offer  from Offer
       where true
       select any  consume 1  gain \{offer.giver\}
 and  accept from Accept
       where accept.id       == offer.id &&
             accept.accepter == offer.receiver
       select any  consume 1  gain \{offer.receiver\}
 and  coin   from Coin
       where coin.issuer     == !Isabelle &&
             coin.holder     == offer.receiver
       select any  consume 1  gain \{!Isabelle, offer.giver\}
to
     say Coin [ issuer = !Isabelle
              , holder = offer.receiver ]
      by  \{!Isabelle,offer.receiver\}  obs \{!Mona\}
      use \{'transfer\}                 num 1
\end{alltt}
\end{small}

\caption{Desugared Coin Transfer Rule}
\label{f:CoinTransferDesugared}
\end{figure}


% -----------------------------------------------------------------------------
\subsection{Grammar}
The grammar for the core language is in Figure~\ref{f:Grammar}. A matching desugared version of the coin transfer rule from Figure~\ref{f:CoinTransfer}, using @!Mona@ as an observer, is in Figure~\ref{f:CoinTransferDesugared}.
A $\iRule$ has a name, pattern matching clauses, and a term for the body to produce a set of new factoids. We use EBNF, so $\iMatch+$ in the production for $Rule$ requires at least one match clause. Each $\iMatch$ clause has form:
$$
X~ @from@~ N~ @where@~ M~ @select@~ C~ @consume@~ U~ @gain@~ I
$$
This says we should scan through all facts in the store with name $N$, binding each in turn to the variable $X$ which is in scope for the sub-terms in the clause. We then \emph{gather} all such facts that satisfy predicate $M$ into a set, \emph{select} the single fact specified by $C$, \emph{consume} the weight specified by $U$, and \emph{gain} the authority specified by $I$. In the select clause $C$, the @any@ keyword indicates that any gathered fact that satisfies the @where@ predicate can be selected. For $@first@~ M$ and $@last@~ M$ we sort facts by the key $M$ and take the first or last one. Our gather/select/consume/gain process is a regularized database query, reminiscent of the FLWOR blocks of XQuery~\cite{Boag2002:XQuery}. Join style queries are expressed using multiple fact matching patterns. We default previously elided @select@, @consume@ and @gain@ clauses to @select any@, @consume 1@ and @gain {}@ respectively.

In the term language we require a few primitive operators to split out the components of a fact value --- @fact'payload@ and so on. In Figure~\ref{f:CoinTransferDesugared}, however, we have left applications of @fact'payload@ implicit for readability, and retained some standard infix operators. The @check@ clauses used in \S\ref{s:Selection} can be desugared into applications of @fact'by@ that check the authority of a fact in a @where@ clause.


% -----------------------------------------------------------------------------
\eject{}
\subsection{Static Semantics}
Our type language is standard. In Figure~\ref{f:Grammar} type ($@Sets@~ Type$) classifies multisets. The type ($@Fact@~ \iType$) classifies fact values whose payload has type $\iType$. The @FACT@ type is used as the supertype of all such fact types, omitting a parameter for the payload type so that rule bodies can produce sets of factoids of differing sorts.

The typing rules are in Figure~\ref{f:Statics}. Most judgment forms use two environments: $\iDecls$ ($\Sigma$) that maps fact names to their payload types, and $\iEnv$ ($\Gamma$) that maps variable names to their types. The grammars for the environments are back in Figure~\ref{f:Grammar}. In the source language example in Figure~\ref{f:CoinTransfer} we specified the payload type of each fact using the @fact@ keyword. In the static semantics here we assume all such types are added to the initial $\iDecls$ environment.

The judgment ($\jOk{\Sigma}{R}$) checks that rule $R$ is well typed. In the premises we check the sequence of pattern matches, producing a type environment $\Gamma$, that lists the types of variables that are in scope in the body of the rule. The body $M$ produces a multiset of new factoids. The judgment ($\jTypeMatch{\Sigma}{\Gamma}{X~ \kfrom~ N~ \kwhere~ M~ ...}{\Gamma'}$) checks a single pattern match, where $X$ is bound to to each fact of name $N$ in turn, and we keep the facts that match the boolean predicate $M$. The premise $(N:T) \in \Sigma$ retrieves the payload type $T$ of the fact, which is used to construct the type of $X$ which is in scope in the rule body $M$. Checking of $\iSelect$, $\iConsume$, $\iGain$ and $\iTerm$ expressions is straightforward.


% -----------------------------------------------------------------------------

% -----------------------------------------------------------------------------
% Dynamic Semantics
% -----------------------------------------------------------------------------


\begin{figure}
$$
\begin{array}{cl}
\fbox{$\jOk{Env}{Rule}$}
\\[2ex]


\ruleI  {    \jTypeMatch{\cdot}{\Sigma}{H^n}{\Gamma}
         \qq \jType{\Gamma}{M}{\kUnit} }
        {\jOk{\Sigma}{\krule~ N~ \kawait~ H^n~ \kto~ M}}
\end{array}
$$

$$
\begin{array}{cl}
\jTypeMatch{\Gamma}{\Sigma}{\cdot}{\Gamma}
\\[1ex]
\ruleI  {   \jTypeMatch{\Gamma}{\Sigma}{H}{\Gamma'}
        \qq \jTypeMatch{\Gamma'}{\Sigma}{H^n}{\Gamma''} }
        {\jTypeMatch{\Gamma}{\Sigma}{H~H^n}{\Gamma''}}
\end{array}
$$

$$
\begin{array}{cl}
\ruleI  {\begin{array}{ll}
            \jTypeGather{\Gamma}{\Sigma}{X}{G}{\Gamma'}
        \\  \jOk{\Gamma'}{T} \quad \jOk{\Gamma'}{C} \quad \jOk{\Gamma'}{I}
         \end{array}
        }
        {\jTypeMatch{\Gamma}{\Sigma}{X~ \kfrom~ G~ T~ C~ I}{\Gamma'}}
\end{array}
$$

$$
\begin{array}{cl}
\ruleI  {    (N : \tau) \in \Sigma
         \quad \Gamma' = \Gamma,~ X : \kFact~ \tau
         \quad \{ \jType{\Gamma'}{M_i}{\kBool} \}^{i \gets 1 .. n}
        }
        {\jTypeGather{\Gamma}{\Sigma}{X}{\kwhen~N~M^n}{\Gamma'}}
\end{array}
$$

$$
\begin{array}{ccc}
% -- Select
        \jOk    {\Gamma}{\kany}
&       \jOk    {\Gamma}{\kretain}
&       \jOk    {\Gamma}{\ksame}
\\[2ex]
        \ruleI  {\jType{\Gamma}{M}{\kNat}}
                {\jOk  {\Gamma}{\kfirst~ M}}
&       \ruleI  {\jType{\Gamma}{M}{\kNat}}
                {\jOk{\Gamma}{\kconsume~ M}}
&       \ruleI  {\jType{\Gamma}{M}{\trm{\kAuth}}}
                {\jOk{\Gamma}{\kgain~ M}}
\end{array}
$$

$$
\begin{array}{cl}
\fbox{$\jTypeAction{Env}{Env}{Term}{Type}$}
\\[2ex]

\jTypeAction{\Gamma}{\Sigma}{\kdone}{\kUnit}

\\[2ex]
\ruleI  {   \jTypeAction{\Gamma}{\Sigma}{M}{\kUnit}
        \qq \jTypeAction{\Gamma}{\Sigma}{M'}{\tau} }
        { \jTypeAction{\Gamma}{\Sigma}{M ~;~ M'}{\tau} }

\\[2ex]
\ruleI  { \begin{array}{ll}
           (N : \tau) \in \Sigma
           \\ \jType{\Gamma}{M_{by}} {\kSet~ \kParty}
            & \jType{\Gamma}{M_{obs}}{\kSet~ \kParty}
           \\ \jType{\Gamma}{M_{use}}{\kSet~ \kSymbol}
            & \jType{\Gamma}{M_{num}}{\kNat~}
          \end{array}
        }
        {\begin{array}{ll}
         \Gamma ~|~ \Sigma ~\vdash
                   & \hspace{-1ex} \ksay~ N_{tag}~ M_{payload}
                \\ & \hspace{-1ex} \kby~  M_{by}  ~~\kobs~ M_{obs} \kuse~ M_{use}~ \knum~ M_{num}
                :: \kUnit
         \end{array}
        }
\end{array}
$$

$$
\begin{array}{cl}
\fbox{$\jType{Env}{Term}{Type}$}
\\[2ex]
\ruleI  { (X : \tau) \in \Gamma }
        {\jType{\Gamma}{X}{\tau}}
\qq
\ruleI  {   \jType{\Gamma}{M}{\tau \to \tau'}
        \qq \jType{\Gamma}{M'}{\tau} }
        { \jType{\Gamma}{M~ M'}{\tau'} }
\qq \dots
\end{array}
$$

\caption{Static Semantics}
\label{f:Statics}
\end{figure}


% -----------------------------------------------------------------------------
\subsection{Dynamic Semantics}
The evaluation rules are in Figure~\ref{f:Dynamics}. We use the abbreviation $\iAuth$ to mean a set of party values that authorize some fact, $\iFacts$ to mean a set of facts and $\iFactoids$ a map of facts to their weights. We use $\iStore$ to also map facts to their weights, but name it differently to hint that this is the current ledger state used for rule evaluation. We use $\iEnv$ to map variable names to their values. In the notation we indicate that a variable stands for a collection by including a superscript that indicates the size of that collection, so $F^n$ would stand for a set of facts with size $n$.

This semantics can be used to both execute rules to produce a transaction as per~\S\ref{s:Transactions}, and also to validate that a transaction is well formed. The semantic rules are non-deterministic. Given a particular store and production rule definition, it may be possible to execute that production rule by matching several different subsets of facts, and the semantic rules do not specify which particular subset to use. When a particular party builds a transaction and submits the views to others, it is up to the submitter to resolve any  non-determinism as they see fit. Rainfall is a \emph{contract system} also in the sense of specifying a range of valid behavior, rather than an abstract machine that fixes a single order for rule evaluation.

In Figure~\ref{f:Dynamics}, starting with the top-level EvFire rule, the judgment: \\
($\jFire{A_{\isub}}{S}{R}{F^r_{\iread}}{D^s_{\ispend}}{D^n_{\inew}}{S'}$) says that a submitting party with authority $A_{\isub}$ and initial store $S$ can execute rule $R$, which reads facts $F^r_{\iread}$, spends factoids $D^s_{\ispend}$, creates new factoids $D^n_{\inew}$, producing a new store $S'$. The result sets can be used to produce (or check) a transaction structure, where the input list in the transaction is formed from both $F^r_{\iread}$ and $D^s_{\ispend}$, using zero valued weights for facts listed in $F^r_{\iread}$ which are read but not consumed.

The semantics is specified from a global point of view, where the $\iStore$ includes the complete set of factoids visible to all parties. The particular subset of facts visible to a submitting party is controlled by $A_{\isub}$, which is the authority of that party. In the premises of EvFire, we apply the pattern matches to produce a set of read and spent facts, along with the gained authority $A_{\igain}$, which must cover the by-authority of all new factoids $D^n_{\inew}$ produced by the body of the rule. The premise $\jStoreDown{S}{D^s_{\ispend}}{S'}$ checks that factoids needing to be spent are available in the store $S$ with sufficient weight, and then removes them from the store, producing a new store $S'$. Similarly $\jStoreUp{S'}{D^n_{\inew}}{S''}$ adds the new facts produced by the rule body. The corresponding rules are in Figure~\ref{f:StoreModification}.

Rules EvMatchNil/Cons apply the pattern matches to the store, gathering the set of facts read factoids spent, authority gained and the environment containing the matched facts. We use $\uplus$ operator to mean multiset union, where the weights of identical facts are summed. Rule EvMatchOne performs a single pattern match, performing the gather/select/consume/gain stages to produce the fact selected, a factoid also mentioning the weight consumed, and the authority gained from that fact. If a fact is to be read but not consumed then the weight $W_{\ispend}$ will be zero. The $(F,~ 0)$ factoid produced by EvMatch will be eliminated by the use of $\uplus$ in EvMatchCons, but the fact $F$ will be retained in the set of facts read.

Rule EvGather collects the facts that match the gather predicate. The premise is written as a set comprehension, where the clause ``$\trm{sees}~ A_{\isub}~ F$'' ensures we only include facts visible to the submitting party. The ``$\trm{sees}$'' predicate was defined back in \S\ref{s:Transactions}.

Importantly, we bind the \emph{fact} value instead of the whole \emph{factoid} to avoid confusion about what weight a pattern match should observe when a pattern before it consumes the same fact. Binding the whole factoid would allow rules to check the weights of factoids more directly than using @consume@ clauses, but then interleaving consumption with matching would mean patterns could not be reordered freely. Similar issues arise in related \emph{active database systems}~\cite{Paton1999:Active}. A programmer would ultimately want to specify further behavior, but we leave this to future work.

Rules EvAny/First specify how a single fact should be selected from the set of gathered facts. With EvAny any fact can be selected. In EvFirst we compute a set of pairs $D^m$ of sort keys and values, and select the value with the smallest key. Handling @last@ is similar.

Rule EvConsumeSome evaluates the term specifying the fact weight to be consumed, and checks the current rule is in the use set of the fact. The EvConsumeNone version does not need the check, as the fact itself is not consumed. Rules EvGainNone/Some are similar, with EvGainSome checking that the fact supply the desired authority before returning it.

A direct implementation of the semantic pattern matching rules would essentially compute a relational join using naive cartesian product and filtering. A production implementation could instead use the RETE algorithm~\cite{Forgy1981:RETE, Doorenbos1995:ProductionMatching}, a parallel extension of it~\cite{Aref1998:LanaMatch}, or by conversion onto relational algebra for execution on a back-end relational database that maintains indexing structures.

Rule EvSay evaluates its arguments and produces the corresponding factoid. The remaining execution rules for the STLC term language are standard and have been omitted to save space.


% -----------------------------------------------------------------------------
% Dynamic Semantics
% -----------------------------------------------------------------------------

\begin{figure*}
$$
\begin{array}{cl}

% -- Fire -----------------------------
\fbox{$\jFire{Auth}{Store}{Rule}{Factoids}{Factoids}{Store}$}
\\[2ex]
\ruleI
{ \begin{array}{ll}
  &  \bigwedge \{ A_{has} \supseteq \trm{auth-by}~D~|~D \in D_{new}^m \}
  \\[1ex] \jMatches{N}{A_{sub}}{S}{\cdot}{H^n}{A_{has}}{D_{spent}^n}{E'}
  &  ~~~ \jExec{E'}{M}{\kunit}{D_{new}^m}
  \end{array}
}
{\jFire {A_{sub}}{S}
        {\krule~N~\kawait~H^n~\kto~M}
        {D_{spent}^n}{D_{new}^m}{(S \setminus D^n_{spent}) \uplus D_{new}^m}
}
& (\trm{EvFire})
\\[4ex]


% -- matches --------------------------
\fbox{$\jMatches{Name}{Auth}{Store}{Env}{Matches}{Auth}{Factoids}{Env}$}
\\[2ex]

\jMatches{N}{A}{S}{E}{\cdot}{\emptyset}{\emptyset}{E}
& (\trm{EvMatchNil})
\\[2ex]

\ruleI
{ \begin{array}{ll}
       \jMatch{N_{rule}}{A_{sub}}{S~}{E~}{H~~}{A_{gain}}{D_{spent}}{S'}{E'}
  \\[0.5ex]
       \jMatches{N_{rule}}{A_{sub}}{S'}{E'}{H^n}{A_{gain}'}{D_{spent}'}{E''~~~~~  }
  \end{array}
}
{ \jMatches{N_{rule}}{A_{sub}}{S}{E}{H ~ H^n}
           {A_{gain}  \cup A_{gain}'}
           {D_{spent} \cup D_{spent}'}
           {E''}
}
& (\trm{EvMatchCons})
\\[4ex]


% -- match ----------------------------
\fbox{$\jMatch{Name}{Auth}{Store}{Env}{Match}{Auth}{Factoid}{Store}{Env}$}
\\[2ex]

% -- MatchGuard
\ruleI
{ \begin{array}{ll}
     \jGather{A_{sub}}{S}{E}{X}{G}{F^n}
  &  ~~ \jGain{F}{E'}{I}{A_{gain}}
  \\ ~~~~~~~~~\jSelect{F^n}{E}{X}{T}{F~~}
  &  ~~ E' = E, X \mapsto F
  \\ ~~~~ \jConsume{F}{S}{E'}{C}{W}{S'}
  &  ~~ N_{rule} \in \trm{rules}~ F
  \end{array}
}
{   \jMatch{N_{rule}}{A_{sub}}{S}{E}{X~\kfrom~G~T~C~I}{A_{gain}}{(F,W)}{S'}{E'}
}
& (\trm{EvMatch})
\\[4ex]


% -- gather ----------------------------
\fbox{$\jGather{Auth}{Store}{Env}{Var}{Gather}{Facts}$}
\\[2ex]

\ruleI
{ F^n = \left \{
  \begin{array}{rll}
        F & |~~ (F \mapsto W) \in S
     \\    & ,~~ \trm{name}~ F = N_{fact}
             ,~~ \trm{sees}~ A_{sub}~ F
             ,~~  W \geq 1
     \\    & ,~~ (\jEval{E,~X \mapsto F}{M^m}{\ktrue})
     \end{array}
  \right \}
}
{   \jGather{A_{sub}}{S}{E}{X}{\kwhen~N_{fact}~M^m}{F^n}
}
& (\trm{EvWhen})
\\[4ex]


%% -- select ---------------------------
\fbox{$\jSelect{Facts}{Env}{Var}{Select}{Fact}$}
\\[2ex]

\ruleI
{ F \in F^n }
{ \jSelect{F^n}{E}{X}{\kany}{F} }
\qq
\ruleI
{ \begin{array}{ll}
      (V', F')  \in D^m
  \qq V' = \trm{minimum}~ \{ V ~|~ (V, \_) \in D^m \}
  \\  ~~~~~~~ D^m \in \{ (V, F) ~|~ F \in F^n, (\jEval{E, X \mapsto F}{M}{V} ) \}
  \end{array}
}
{ \jSelect{F^n}{E}{X}{\kfirst~ M}{F} }
& (\trm{EvAny/First})
\\[4ex]


%% -- consume --------------------------
\fbox{$\jConsume{Fact}{Store}{Env}{Consume}{Weight}{Store}$}
\\[2ex]

 \jConsume{F}{S}{E}{\kretain}{0}{S}
 & (\trm{EvRetain})
\\[2ex]

\ruleI
{     \jEval{E}{M}{W_{need}}
  \qq W_{avail} \geq W_{need}
}
{ \jConsume
        {F}{S[F \mapsto W_{avail}]}{E}
        {\kconsume~ M}
        {W_{need}}{S[F \mapsto W_{avail} - W_{need}]}
}
& (\trm{EvConsume})
\\[4ex]


%% -- gain ----------------------------
\fbox{$\jGain{Fact}{Env}{Gain}{Auth}$}
\\[2ex]
\jGain{F}{E}{\ksame}{\emptyset}
\qq
\ruleI
{     \jEval{E}{M}{A}
  \qq A \subseteq \trm{auth-by}~ F
}
{ \jGain{F}{E}{\kgain~ M}{A}
}
& (\trm{EvSame/Gain})

\end{array}
$$

\caption{Dynamic Semantics}
\label{f:Dynamics}
\end{figure*}



% -----------------------------------------------------------------------------
\eject{}
\subsection{Properties}
\label{s:Properties}
We have mechanized the Rainfall semantics using the Isabelle/HOL interactive theorem prover, and proved several useful isolation and authorization properties. Using this theory as a basis, we have also mechanized the market example from \S\ref{s:Selection} to demonstrate how to prove safety properties of business logic encoded in our system. The proof scripts and full statements of invariants are available at
\url{http://github.com/rainfall-lang}.


% -------------------------------------
\subsubsection{Properties of the Semantics}
\label{s:PropertiesOfSemantics}
\begin{theorem}
Frame Constriction:
Given an arbitrary store, if a rule can fire using facts from that store, producing a set of facts read, factoids spent, and new factoids created, then the same rule can fire in a store containing only information from those produced sets.
\end{theorem}
\begin{tabbing}
MMx \= MMMMMM \= MM \kill
\trm{If}     \> $\jFire{A}{S}
                       {R}{F_{\iread}^r}{D_{\ispent}^n}{D_{\inew}^m}{S'}$ \\
\trm{then}   \> $\jFire{A}{F_{\iread}^r \Cup D_{\ispent}^n}
                       {R}{F_{\iread}^r}{D_{\ispent}^n}{D_{\inew}^m}{S''}$ \\
             \> \trm{where} $S'' ~=~ D_{\inew}^m \uplus (F_{\iread}^r - D_{\ispent}^n)$
\end{tabbing}

Frame Constriction guarantees that the (unblinded) transaction structures described in \S\ref{s:Transactions} can always be validated in isolation, without needing the entire state of the ledger that existed at the time the transaction was formed.

In our formalization we use multisets for both the set of facts read $F^r_{\iread}$ and factoids spent $D^n_{\ispent}$. When forming the store used for the second rule firing we need to ensure that facts listed in both $F^r_{\iread}$ and $D^n_{\ispent}$ are not counted twice. To achieve this we use the $\Cup$ operator which performs a version of multiset union where elements that exist in both of the argument sets are given their maximum weight in the result set, rather than their sum.


% -------------------------------------
\begin{theorem}
Focused Firing: Facts that are not visible to a submitting party do not influence rule firing:
\end{theorem}
\begin{tabbing}
M \= MMx \= MMMMM \= MMM \kill
\> If     \> $\jFire{A_{\isub}}{S}{R}{F_{\iread}^r}{D_{\ispent}^n}{D_{\inew}^m}{S'}$ \\
\> ~and   \> $\forall f \in S_{\iothers}.\ \neg (\trm{sees}~ A_{\isub}~ f)$ \\
\> then   \> $\jFire{A_{\isub}}{S \uplus S_{\iothers}}{R}{F_{\iread}^r}{D_{\ispent}^n}{D_{\inew}^m}{S'}$
\end{tabbing}


% -------------------------------------
\begin{theorem}
Visible Spending: A rule firing cannot influence facts that are not visible to the submitting party.
\end{theorem}
\begin{tabbing}
MMM \= MMx \= MMMMMMM \= MM \kill
\> If     \> $\jFire{A_{\isub}}{S}{R}{F_{\iread}^r}{D_{\ispent}^n}{D_{\inew}^m}{S'}$ \\
\> then   \> $\forall f \in F_{\iread}^r \Cup D_{\ispent}^n.\ \trm{sees}~ A_{\isub}~ f$
\end{tabbing}

Focused Firing ensures that the validity of transaction views sent by the submitter to a receiving party will not be influenced by extra facts that are visible to the receiver but not the submitter. Dually, Visible Spending ensures that any extra facts that are visible to the receiver but not the submitter cannot be influenced by those views.


% -------------------------------------
\begin{theorem}
Authority Flow: If a fact created by a rule firing is authorized by some party, then the same party authorized a fact that was read or spent by that rule firing.
\end{theorem}
\begin{tabbing}
M \= MMx \= MMMMMMM \kill
\> If   \> $\jFire{A_{\isub}}{S}{R}{F_{\iread}^r}{D_{\ispent}^n}{D_{\inew}^m}{S'}$ \\
\> then \> $\forall f \in D_{\inew}^m.\
            \forall a \in \trm{auth-by}~f.\
            \exists d \in F_{\iread}^r \Cup D_{\ispent}^n.$ \\
\>      \> \hspace{1em}$a \in \trm{auth-by}~d$
\end{tabbing}

This key property of our authorization system implies that the submitter of a transaction cannot attach their own authority to any of the created facts. The authorization of facts is controlled by the production rules expressed in the system, rather than the particular parties that form transactions.


\eject{}
% -----------------------------------------------------------------------------
\subsubsection{Properties of the market example}
We have also mechanized the market example in \S\ref{s:Selection} in Isabelle/HOL and proven the following business level theorem:

\begin{theorem}
Budget Adherence: The total value of invoices a broker receives for items bought on behalf of a client never exceeds the budget specified in the client's original order.
\end{theorem}

We prove this by first specifying an invariant over all facts in the store, and then proving that the possible firings of each rule in the workflow all preserve the invariant. For example, the top-level statement for the @reserve@ production rule is:
\begin{tabbing}
M \= MMx \= MMMMMM \kill
\> If   \> $\jFire{A_{\isub}}{S}{@reserve@}{F_{\iread}^r}{D_{\ispent}^n}{D_{\inew}^m}{S'}$ \\
\> ~and \> $\trm{store-ok}~{S}$ \hspace{1ex} then \hspace{1ex} $\trm{store-ok}~{S'}$
\end{tabbing}

The statement of the invariant is available in the appendix. We also prove that the invariant is established for newly created budgets with no initial associated bids, invoices or offers.

Rainfall's production rule based programming model makes it easy to approach proofs of business level properties like this. Provided one can write down a suitable invariant, the task then decomposes automatically into a separate subgoal for each rule.


% -----------------------------------------------------------------------------
\subsection{Extensions}

\begin{figure}
$$
\begin{array}{cc}
%% -- store
\fbox{$\jStoreDown{\iStore}{\iFactoids}{\iStore}$}
&
\fbox{$\jStoreUp  {\iStore}{\iFactoids}{\iStore}$}
\\[3ex]
\ruleI  {       D^n \subseteq S
        \quad   S' = S - D^n}
        {\jStoreDown{S}{D^n}{S'}}
&
\ruleI  {       S' = S \uplus D^n}
        {\jStoreUp{S}{D^n}{S'}}
\end{array}
$$
\vspace{-1em}
\caption{Store Modification}
\label{f:StoreModification}
\end{figure}


% -----------------------------------------------------------------------------
The semantics described in this section has a few natural extensions that we have elected to omit to avoid obscuring the presentation.

\subsubsection{Extended Weight Types}
There is no particular reason why the weight of each fact must be restricted to being a natural number as per~\S\ref{s:Weights}. Our dynamic semantics only depends on the representation of weights in three places: to combine the weights of factoids in EvMatchCons and EvStoreUp, and in the subset test and set difference operators in EvStoreDown.

In general it would sufficient to use an Abelian group which also has a partial order. An Abelian group provides a weight combining operator that is associative, commutative and invertible. Requiring weight combining operator to be associative and commutative allows the facts needed by a rule to be matched in arbitrary order. We need the partial order to determine if a weight of the required value is present in the store, and the weight combining operator needs to be invertable so we can reduce the weight of facts in the store when they are spent.

An example extended weight type is a simple @Present@ / @Absent@ indicator. When multiple facts are added with a weight of @Present@ then the resulting fact is still @Present@. If this @Present@ fact is then consumed it becomes @Absent@. Such a weight type helps to encode facts like ``someone needs to water the plants''. As the plants should only be watered once, it does not make sense to allow a numeric weight with value greater than one. Later, when a real-world party actually gets around to watering the plants, then the whole fact can be removed from the store.


% -----------------------------------------------------------------------------
\subsubsection{Minimum Weight Thresholds}
In EvFire from Figure~\ref{f:Dynamics}, when a production rule reads a fact but does not consume any weight of it, or gain any authority from it, then the fact appears in the set of facts read $F^r_{read}$, but not the map of factoids spent $D^s_{spend}$. This means the rule can fire if the required fact is present in the store with any non-zero weight. An extension is to allow a rule to wait for a particular weight of a fact to be available before firing, without also needing to consume it. To achieve this we would change the representation of $F^r_{read}$ to also be a map of factoids, and add a (@require@ M) form to $Consume$ specifications, where M is a term that evaluates to the weight the rule must have before firing.

