
\clearpage{}


\begin{figure}
\begin{small}
\begin{alltt}
fact  Coin   [issuer: Party,  holder:   Party]
fact  Offer  [id:     Symbol, terms:    Text,
              giver:  Party,  receiver: Party ]
fact  Accept [id:     Symbol, accepter: Party]

rule  transfer
await Accept [id = ?i, accepter = ?a]            gain \{a\}
  and Offer  [id = i,  giver = ?g, receiver = a] gain \{g\}
  and Coin   [issuer = !Isabelle,  holder = g]
      gain \{!Isabelle,g\}
to
  say Coin   [issuer = !Isabelle,  holder = a]
   by \{!Isabelle, a\} use \{transfer\}
\end{alltt}
\end{small}
\caption{Coin Transfer Workflow}
\label{f:CoinTransfer}
\end{figure}


% -------------------------------------------------------------------------------------------------
\section{Facts, Rules, and Authority}
\label{s:FactsWeights}
The Rainfall programming model includes a ledger of facts and a set of rules. Parties using the system add facts to the ledger, cryptographically signing them to demonstrate that they authorize their contents. Rules match on several facts and create new facts, possibly consuming matched facts in the process. Rules can also gain authority from matched facts, and new facts created can be given a subset of the authority gained from the matched facts. The set of facts visible to each party is also specified by the authority system, so not all facts need to be visible to all parties. In this section we describe facts, rules and the authority system, finishing with the formal definition of the data model.


% -----------------------------------------------------------------------------
\subsection{Facts}
\label{s:Facts}
Figure~\ref{f:CoinTransfer} contains the fact and rule definitions for a simple coin transfer workflow. A @fact@ declaration gives the \emph{tag} and \emph{payload} types of each sort of fact used in the workflow. In this example, a @Coin@ fact represents a single coin that has been created by an @issuer@ party, and is currently held by a @holder@. An @Offer@ fact represents an offer by the coin holder, the @giver@, to transfer their coin to a @receiver@. The offer includes a value of abstract @Symbol@, type that uniquely identifies the offer, and a text string describing the terms of the offer. An @Accept@ fact specifies that the receiver does indeed wish to accept a coin offer with the given terms.

For example, we suppose we have the following facts:
\begin{small}
\begin{code}
 Coin   [issuer = !Isabelle, holder = !Alice]
 Offer  [id     = '1234,  terms    = "To purchase a guitar"
         giver  = !Alice, receiver = !Bob]
\end{code}
\end{small}

Names prefixed by @!@ are literal identifiers of the parties using the system, and their values have type @Party@. Names prefixed by @'@ are symbolic identifiers (strings), and their values have type @Symbol@. The facts reveal that @Alice@ wishes to transfer her coin to @Bob@ for the purchase of a guitar. If @Bob@ wishes to accept her offer he can add the following fact to the ledger:
\begin{small}
\begin{code}
 Accept [id = '1234, accepter = !Bob]
\end{code}
\end{small}

Given the @Coin@, @Offer@, and @Accept@ facts, the @transfer@ rule from Figure~\ref{f:CoinTransfer} can fire, which \emph{consumes} the three input facts and produces a new one:
\begin{small}
\begin{code}
 Coin   [issuer = !Isabelle, holder = !Bob]
\end{code}
\end{small}

This new coin belongs to Bob, alternatively, we could say that the coin Alice had has been transferred to Bob.

% CUT: In financial terms, this workflow is a \emph{bilateral transfer} because the receiver must explicitly accept the offered coin, rather than being an \emph{unilateral transfer}, where the receiver has no say in whether the transfer takes place.


% -----------------------------------------------------------------------------
\subsection{Weights}
\label{s:Weights}
Suppose Bob already had a coin, and then receives another one. We handle this by giving each fact a \emph{weight}, which specifies an positive integral number of ``copies''. We indicate the weight of a fact using the @num@ keyword, typically eliding it the weight is one.

For example, suppose the ledger already contained:
\begin{small}
\begin{code}
 Coin [issuer = !Isabelle, holder = !Bob]  num 5
\end{code}
\end{small}
%
When Alice transfers an additional coin to Bob the entry on the ledger would become:
\begin{small}
\begin{code}
 Coin [issuer = !Isabelle, holder = !Bob]  num 6
\end{code}
\end{small}
%
Rules can also consume an arbitrary weight of a fact, including zero which non-destructively reads it, which we discuss in \REF.

% TODO: discuss extension of using general monoids for weights, not just natural number and addition,
%       if we have this then we'd also need to put constraints on the weights, like >= 0.


% -----------------------------------------------------------------------------
\subsection{Rules}
The @transfer@ rule of Figure~\ref{f:CoinTransfer} specifies how existing facts can be combined to create new facts. The rule is written in a syntax based on production rule systems such as OPS5~\cite{Forgy1981:OPS5} and CLIPS~\cite{Riley2017:CLIPS}. The rule definition has the form (@rule@ \emph{name} @await@ \emph{pattern} @to@ \emph{body}) where \emph{pattern} specifies which facts must be matched for the rule to fire, and \emph{body} is a pure term expression that computes the set of new facts to add to the ledger. These new facts can be authorized by a subset of the parties specified by the @gain@ clauses on the patterns.

Names prefixed by @?@ are binding occurrences of variables, so the rule requires an @Accept@ fact with its @id@ field set to some value, binds it to @i@, and must wait for an @Offer@ fact whose @id@ field is set to the same value. The runtime intuition is that matching of facts proceeds in sequence, so the rule will wait for an @Accept@ fact, then an @Offer@ fact, then a @Coin@ fact. Variables bound in earlier facts are in scope when matching latter facts, and also in the body. Implementations of traditional production rule engines based on the RETE~\cite{Forgy1981:RETE} algorithm do not require facts to be matched in order, but we fix the sequence here to simplify the operational semantics.

By default, when a rule matches its required facts those facts are consumed from the active ledger state, and any new facts created in an atomic transaction. We also refer to the process of consuming a fact as \emph{spending} that fact, after the original Unspent Transaction Output (UTxO) model of the Bitcoin system~\cite{Zahnentferner2018:UTxO}.

% CUT: When a fact is spent we say it has become \emph{inactive}, and can no longer be matched by a subsequent rule.



% -----------------------------------------------------------------------------
\subsection{Authority}
\label{s:FactAuthority}
A given currency can only retain value when it is \emph{scarce}. Commodity currencies like gold and silver are scarce due to the physical difficulty of digging them out of the ground. Fiat currencies like Icelandic Kr\'ona (ISK) are scarce because a central authority issues a limited number per year. Bitcoins are scarce because they represent the solution of a particular cryptographic problem, which at a particular time, required significant energy to solve.

In our coin example, the scarcity of coin facts is enforced by requiring that they are authorized by an issuing party @!Isabelle@. We assume that everyone using the system trusts @!Isabelle@ to not add new, signed @Coin@ facts to the ledger in an inappropriate way.\footnote{meaning they will not print money, or at least, not too much} In a private financial system @!Isabelle@ might represent a bank. In a public ledger system, a fixed supply of coins might be generated using a secure multiparty protocol to sign the facts similarly to how the ZCash system~\cite{Bowe2018:MultiParty, Hopwood2016:zcash} was initialized.

In our current example, when coin facts are created we assume they are authorized by both the issuer and the initial holder of the coins. The set of parties which authorize a fact is called the \emph{by-authority set}. Any party can add any fact to the ledger at any time, provided the fact has only that parties authority. Likewise, any party can consume (spend) a fact from the ledger at any time, provided the fact only carried that parties authority. Ensuring that coin facts are always authorized by two parties ensures that neither can unilaterally create, transfer or consume them. Coins facts can only be modified by pre-agreed rules that first collect the authority of all relevant parties.


% -----------------------------------------------------------------------------
\subsection{Observation}
\label{s:Observation}
% TODO: as an extension could make it so rule can only see the other parties that can observe a fact if it also has authority to spend a fact.
% This would need to be handled by the submitter of the transaction.
When a fact carries the authority of a particular party then it is natural that the party should be aware of the fact. In Rainfall, all facts with the authority of a particular party are visible to that party. For example, the original @Coin@ fact from \S\ref{s:Facts} carries the authority of both @!Isabelle@ and @!Alice@, so those parties will be informed of the creation and subsequent consumption of that fact.

% TODO: The obs authority is not "metadata", it is part of the fact, so is real live data.
In practical commercial workflows, it is often the case that extra parties must be informed of events such as coin transfers and contract completion, even though they do not have direct authority over the facts themselves. For example, in Australia, the Australian Securities and Investments Commission (ASIC) monitors the activity of the Australian Securities eXchange (ASX), even though they are not themselves a giver or receiver on any of the currency transfers or securities transactions. In Rainfall, we use an additional meta-data field associated with each fact, called the \emph{obs-authority} set to collect the set of parties that can observe a fact but do not have authority over it. The obs-authority set of a fact may contain the name of a party that is also in the by-authority set for the same fact, though it provides no additional benefit. Privacy will be discussed in more detail in \S\ref{s:Privacy}


% -----------------------------------------------------------------------------
\subsection{Usable Rules}
To conserve the overall number of coin facts we must each rule that adds a coin fact also consumes an identical number of coin facts. As we allow the body of a rule to contain arbitrary code, deciding whether or not an arbitrary rule preserves the quantify of some fact is undecidable in general. Instead of trying to specify an (inevitably incomplete) automated analysis we instead require the parties using the system to ensure the rules use had the desired properties themselves. Each fact is then annotated with the cryptographic hash of the rules that are permitted to consume, and gain authority from it. This final piece of meta-data called the \emph{valid rules set}.

The valid rules set for a fact determines the business-level meaning of the fact. In our current example, the coin facts have the name of the single @transfer@ rule in their valid rules set, meaning that coins are things that can be transferred, but nothing else. As we mentioned in \S\ref{s:Authority} we also arrange coin facts to always be authorized by two separate parties, the issuer and the holder. This means that no single party can unilaterally create or remove coin facts, but they can come to an agreement that through the use of the system, the total number of coin facts that have this special form will be conserved.


% -----------------------------------------------------------------------------
% TODO: say that in a practical system we would store hashes of parties and rule
% sets, not naively list out every element of every set, as replicating this data
% for every fact is likely not a good use of space.
\subsection{Now with Metadata}
\label{s:NowWithMetadata}

Returning to our coin transfer example from \S\ref{s:Facts}, we restate the facts that will be consumed, along with all their meta data.

\begin{small}
\begin{code}
 Accept [id = '1234, accepter = !Bob]
    by  {!Bob}                obs {!Mona, !Alice}
    use {'transfer}           num 1

 Offer  [id = '1234, terms = "To purchase one Guitar"
         giver = !Alice, receiver = !Bob]
    by  {!Alice}              obs {!Mona, !Bob}
    use {'transfer}           num 1

 Coin   [issuer = !Isabelle, holder  = !Alice]
    by  {!Isabelle, !Alice}   obs {!Mona}
    use {'transfer}           num 100
\end{code}
\end{small}

In summary, the @Accept@ in the first fact is the \emph{tag} of the fact, and the following record the \emph{payload}. The @by@ field specifies the \emph{by-authority set}, which contains the parties that the fact has been authorized by. The @obs@ field specifies the \emph{obs-authority set}, which contains the extra parties that can observe the fact but do not necessarily authorize it. The @use@ field gives the \emph{rules set}, which are the names of the rules that consume a fact and gain authority from it. Finally, the @num@ field gives the \emph{weight} of the fact, which can be interpreted as the number of usable ``copies'' of the fact.

When the rule from Figure~\ref{f:CoinTransfer} matches on each of these facts in turn, it gains the authority specified by the @gain@ clause in the pattern match. For the above facts, the first pattern gains authority of @!Bob@, and the third the authority of @!Isabelle@ and @!Alice@. When all three patterns have matched the rule \emph{fires} and executes the body with all the matched variables from @?i@, @?a@, @?r@ and so on in scope. In this case the body uses the @say@ clause to produce a new fact, giving it the authority of the issuer @!Isabelle@ and new holder @!Bob@. Note that we do not need a @gain g@ clause in the second pattern, because if @Alice@ does not want to give away her coin then she can just not add an @Offer@ fact to the system. The @Offer@ fact can be consumed by virtue of the fact that it has @transfer@ in its valid rules set. The rules needs to gain the authority of @Isabelle@ and @Bob@ to create the new @Coin@ fact, but as no new fact is created with the authority of @Alice@, it does not need hers as well.

We are now in a position to describe the full data model of the ledger, which appears below. The current ledger state is a map from $Fact$ to its current $Weight$, where the $Fact$ includes the \emph{by-authority}, \emph{obs-authority} and \emph{rules-set} along with the tag and payload. If two facts have different authority meta-data then they are taken to be different facts.
$$
\begin{array}{ll}
   State   & = Map~ Fact~ Weight
\\ Fact    & = (Name, Record, By, Obs, Rules)
\\ Record  & = List~ (Name, Value)
\\ By      & = Set~ Party
\\ Obs     & = Set~ Party
\\ Rules   & = Set~ Name
\\ Weight  & = Nat
\end{array}
$$
If a given fact does not appear in the set then we assume that the weight is zero. The $State$ here is the current state of the ledger. As we will discuss in \REF the \emph{ledger} itself is a list of transactions, that includes the full history of all changes, whereas a $State$ refers to the facts that are active after the most recent transaction.

% In this semantic model we specify the parties that authorize a given fact just by including their names in the appropriate set of the fact tuple. We discuss the concrete implementation details of cryptographic signing and hashing in \REF.


% -----------------------------------------------------------------------------
\subsection{Ledger Integrity}
As mentioned in the previous section, the new facts created by a rule can be authorized by a set of parties that is at most the parties whose authority has been gained by matching on existing facts. This restriction allows Rainfall to operate as an open system, where arbitrary new parties can join, and create whatever new facts and rules they see fit. In doing so, it is not possible for a new party to cause a fact to be created that is carries the authority of another party that did not explicitly agree to its creation. For example, consider the following rule:

\begin{small}
\begin{code}
  rule  mint
  await Mint [minter = ?m] gain m
   to   say Coin [issuer = 'Isabelle, holder = m]
         by {'Isabelle, m} use {transfer}
\end{code}
\end{small}

An arbitrary party can add a @Mint@ fact to the system to trigger this rule, and state that they should be the holder of the new coin. However, there is no way for the rule to gain the authority of @Isabelle@, the usual coin issuer. As we will see in \REF our operational semantics performs a runtime check to ensure that the authority of all new facts created by a rule are covered by the authority that has been gained by matching against existing ones.

