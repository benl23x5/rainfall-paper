
\clearpage{}

\label{s:Related}
\section{Related Work}

% ---------------------------------------------------------
\subsection{Inference Engines}
Production rule engine OPS~\cite{Forgy1981:OPS5} latter versions CLIPS~\cite{Riley2017:CLIPS} and Drools~\cite{Proctor2008:Drools}. Implementations based on the RETE~\cite{Forgy1981:RETE} algorithm and enhancements \cite{Doorenbos1995:ProductionMatching} of it. Our system is similar to a production rule engine, but with the facts annotated by sets of parties and other authority metadata. In systems like OPS the facts are part of a global set. In our system we annotate facts with sets of parties that control visiblity.


% ---------------------------------------------------------
\subsection{Account Based Ledgers}
In public account ledgers like Ethereum~\cite{Wood2014:Ethereum} and EOS.IO~\cite{Lee2018:EOSIO} all data on the ledger is visible to all parties. In these systems each account is associated with a contract that has its own mutable state. The contract code can export methods which are callable by other parties in the system. Authority is managed programatically, where method code must check at runtime that the party that calls a particular method is permitted to do so. Contracts that require authority from multiple parties to perform some action can use local mutable state to build a set of parties that have provided their consent so far. In our system the authority to create and consume facts is managed at the meta level as this also affects the whether each party can see the data. Rules are free to perform their own ``business level'' checks, such as ensuring that only a given set of parties can activate the rule.


% ---------------------------------------------------------
\subsection{UTXO Based Ledgers}


% ---------------------------------------------------------
\subsection{Petri-net Systems}



% ---------------------------------------------------------
\subsection{Linda-style Tuple Spaces}
Linda~\cite{Gelernter1985:Linda} is a CSP-like process language where the individual processes communicate by adding, removing and non-destructively reading tuples from a globally shared tuple space. Tuples have named fields, and retrieved using simple, typed patterns. For example, we could retrieve (input) a tuple that specifies the price of apples with the following pattern:
\begin{code}
   in(name("Apple"), price(?p))
\end{code}
Processes that try to retrieve a tuple that matches some pattern block until one is available. If multiple tuples match a given pattern then one is selected nondeterministically.

The basic Linda communication model is open, meaning that any party connected to it is free to add and remove tuples at will. The lack of access control or provenance information on data makes it unusable for as a communication medium for mutually distrusting parties. For example, given a tuple:
\begin{code}
   (giver("Alice"), receiver("Bob"), amount(100))
\end{code}
There is no guarantee that Alice was the one that added this tuple to the system, and there is nothing preventing a third party from removing before the workflow completes and Bob gets his money.

\subsection{Law Governed Linda}
Law Governed Linda~(LGL)~\cite{Minsky1994:LawGovernedLinda} takes the basic Linda model and inserts a \emph{controller} between the tuple space and each communicating process. Each controller has a copy of a communication law, written in a fragment of Prolog, that specifies the allowable interactions with the tuple space. For example, the law may specify that a process can only add a tuple such as the above to the space, when the @giver@ field is set to the identify of the process that adds it. In this way, the codified law specifies the allowable \emph{interaction} with the communication medium.

In contrast, the rules in our Rainfall model do not seek to limit the form of data added to the system. Instead, they specify how authorized facts that are already in the system may be combined to generate new authorized facts. In Rainfall there is nothing stopping a party from adding a fact to the system which says ``I have a million dollars", just with their own authority. The key property we need to manage is whether anyone else should rightly believe them. Recall that in the coin transfer example from \S\ref{s:Facts}, the coin facts always carry the authority of multiple parties, which is justified by the sequence of transactions that result in their production. The coin issuer (@Isabelle@ in the example) initially creates, specifies the rules for using them, and signs the facts with their own authority. When the coins are transferred onwards the recorded transactions are also signed by the parties that propose them. Assuming most parties in the system are honest, invalid transactions will simply not be confirmed. Dishonest parties are free to corrupt their own databases by adding invalid transactions, but no one else using the system will accept them.

\subsection{SecSpaces and Scopes}
SecSpaces~\cite{Busi2003:SecSpaces} and Scopes.






% ---------------------------------------------------------
\subsection{Business Process Execution Language}
Business Process Execution Language (BPEL)~\cite{Andrews2003:BPEL} itself lacks a formal semantics. Various efforts to assign semantics, eg using Petri Nets~\cite{Lohmann2009:PetriBPEL} and Abstract State Machines \cite{Fahland2005:SemanticsBPEL}. Much of the effort is in managing compensating transactions~\cite{Colombo2011:Compensating} which are a higher level language concern.


% ---------------------------------------------------------
\subsection{Process Algebras}
The Join Calculus~\cite{Cedric1996:Reflexive} and Chemical Abstract Machine~\cite{Berry1992:Chemical} Process algebras that model local interaction of code and data. CHAM has very explicit rules about grouping (cooling) and ungrouping (heating) facts to form compounds. For our application we want to model locality but it's sufficient to use a seggregated global ``top level'' store, rather than than explicit nesting of stores.


% ---------------------------------------------------------
\subsection{Authorization Logics}
The Dependency Core Calculus (DCC)~\cite{Abadi1999:DCC} extends Moggi's computational lambda calculus with an extra judgment to say that a value produced from a computation is protected at a given security level. The computation types are written (T l a) where T is the computation type former, l an element from a lattice describing authority levels, and a the return type. The calculus has rules for monadic 'lift' and 'bind' operators to ensure results computed by bind are annotated with the required authority. The terms are given a denotational semantics, but not an operational one that dynamically checks the running term has the required authority.

Abadi 2006~\cite{Abadi2007:AccessControl} studies the use of DCC for access control and tracking in a distributed system. It uses a proposition (A says s) where 'A' is some principle/party that affirms statement 's'. The 'says' former abstracts away from the precise details of what is being authenticated or authorized. The statement (A says s) can be interpreted variously as "A has caused s to be said", "s has been said on A's behalf" or "A supports s". This work uses polymorphism to introduce delegation, providn terms of type (forall X. (A says X) implies (B says X)). As per the previous work \cite{Abadi1999:DCC} the system is presented as a logic rather than a programming language with operational meaning.

Garg 2006~\cite{Garg2006:Constructive} presents a system in sequent style with two judgement forms (A true) and (K affirms A), where K is a principle (party) and A is a proposition. The 'affirms' judgement is then internalised into a proposition as (K says A). The meanings of the connectives are defined by the rules that introduce and eliminate them, rather than a concrete operationa semantics. This system comes with meta theory of affirmation flow, meaning that unless a principle K affirms a statement, no affirmations of the form (K affirms A) can be derived from it.

Bowers 2007~\cite{Bowers2007:Consumable} contains a Gentzen style presentation of Garg 2006 that adds a linear environment to enforce that particular statements (K says A) are only used once in a given authority proof. It also has a (A signed F) form to model a message being cryptograpically signed. Includes an informal discussion of how linear uses of predicates can be enforced in a distributed environment by reling on a central ratifier to check the proofs.

Aura~\cite{Jia2008:Aura} is a dependently typed language that includes the (A says P) form of DCC~\cite{Abadi1999:DCC} and provides a proof term assignment for the earlier pure authorization logics. The proof language is restricted to only contain proofs with no redexes, which simplifies type checking and ensures proof construction code does not perform effects. Proofs of authority are passed as pure proof terms so that fact that a given function may capture authority in its closure is not visible in its type.


% ---------------------------------------------------------
\subsection{Contract Pricing Systems}
