

\appendix
% -----------------------------------------------------------------------------
\section{Market Rules}
This section includes the full rule definitions of the market example that appears in \S\ref{s:Query}.
An executable version is available online.

\begin{small}
\begin{code}

fact Order   [desc: Text, limit: Nat,  budget: Nat]
fact Item    [lot:  Nat,  desc:  Text, ask: Nat]
fact Accept  [lot:  Nat,  price: Nat]
fact Offer   [lot:  Nat,  price: Nat]
fact Bid     [lot:  Nat,  offer: Nat]
fact Budget  [desc: Text, total: Nat, remain: Nat]
fact Reserve [lot:  Nat,  bid:   Nat]
fact Invoice [seller: Party, buyer: Party, desc: Text, amount: Nat]


-- Brendan reserves a portion of the budget allotted to
-- Alice, and forwards a bid to Mark.
rule  reserve
await Order     [desc  = ?d, limit = ?l]
      consume none                       gain  {!Alice}
 and  Item      [lot   = ?o, desc = d, ask = ?a]
      select first a  consume none       check {!Mark}
 and  Budget    [desc  = ?d, total = ?t, remain = ?m]
      gain  {!Brendan}
 and  Reserve   [lot   = o,  bid    = ?b]
      where b <= l && b <= a && b <= m   gain {!Brendan}
 to union
      (say Budget [desc = d, total = t, remain = m - a]
       by {!Brendan} use {'reserve})
      (say Bid    [lot   = o, offer = b]
       by {!Brendan, !Alice} obs {!Mark} use {'bid})


-- Mark converts bids that Brendan has placed that are below
-- the asking price of the item into resting offers.
rule  bid
await Bid    [lot = ?o, offer = ?b]   gain {!Brendan}
  and Item   [lot = o,  ask   = ?a]
      where b < a  consume none       gain {!Mark}
 to
      say Offer [lot = o, price = b]
      by {!Brendan, !Mark} use {'accept}


-- Mark accepts a resting offer, removes the item listing
-- and produces an invoice for the sale price.
rule  accept
await Accept [lot = ?o, price = ?p]  gain {!Mark}
  and Offer  [lot = o,  price = p]   gain {!Brendan, !Mark}
  and Item   [lot = o,  desc  = ?d]
      consume 1                      check {!Mark}
 to
      say Invoice [ seller = !Mark, buyer = !Brendan
                  , desc = d, amount = p]
      by  {!Mark, !Brendan}






\end{code}
\end{small}


% -----------------------------------------------------------------------------
\eject
\section{Market Invariants}

The market example introduced in~\S\ref{s:Selection} used the ``\trm{store-ok}'' invariant to ensure that, if the budgets are adhered to in a particular store, then after execution of any of the market rules, any updated budgets in the new store are also adhered to.
A budget is adhered to when the total value of invoices issued to a broker for a particular client do not exceed the client's specified budget limit.

\begin{tabbing}
M \= For all rules \= MMMMMM \kill
\> For all rules \> $r \in \{@accept@, @bid@, @reserve@\}$, \\
\> ~if   \> $\jFire{A_{sub}}{S}{r}{F_{read}^r}{D_{spent}^n}{D_{new}^m}{S'}$ \\
\> ~and \> $\trm{store-ok}~{S}$ \hspace{1ex} then \hspace{1ex} $\trm{store-ok}~{S'}$
\end{tabbing}

The definition of the invariant for stores finds all orders in the store, and ensures that for each order the order invariants are satisfied:

\begin{tabbing}
M \= M \= For all orders \= MMMMMM \kill
\> $\trm{store-ok}~{S} = $ \\
\> \> For all orders \> $@Order@~o \in S$, \\
\> \> require \> $\trm{order-ok}~{S}~o$
\end{tabbing}

The order invariant ensures that an order has at most one budget, and that the budget invariants are satisfied. An order with no associated budget is valid:

\begin{tabbing}
M \= M \= For all budgets \= MMMMMM \kill
\> $\trm{order-ok}~{S}~o = $ \\
\> \> For all budgets \> $@Budget@~b \in \trm{budgets-for-order}~S~o$, \\
\> \> require \> $\trm{unique}~S~b$ \\
\> \> and require \> $\trm{budget-ok}~{S}~b$
\end{tabbing}

While our desired property is to show that the total value of invoices does not exceed the budget limit, our invariant must show a stronger property, which is that the total value of bids, invoices and offers must not exceed the budget limit.
This stronger property is required as bids and offers can eventually result in invoices, as bids are transformed to offers, and offers are accepted.
The budget invariant finds all associated bids, invoices and offers, and sums their prices to compute the total amount reserved by the budget.
This total reserved amount plus the remaining budget must equal the budget limit:

\begin{tabbing}
M \= M \= Require \= invoices \= \kill
\> $\trm{budget-ok}~{S}~b = $ \\
\> \> Require \> $\trm{budget-total}~b = \mathit{total} + \trm{budget-remain}~b$ \\
\> \> where \> $\mathit{total}$ \> $= \mathit{bids} + \mathit{invoices} + \mathit{offers}$, \\
\> \> and \> $\mathit{bids}$ \> $= \sum_{d \in \trm{bids-for-budget}~S~b} \trm{bid-price}~d$ \\
\> \> and \> $\mathit{invoices}$ \> $= \sum_{i \in \trm{invoices-for-budget}~S~b} \trm{invoice-price}~i$ \\
\> \> and \> $\mathit{offers}$ \> $= \sum_{o \in \trm{offers-for-budget}~S~b} \trm{offer-price}~o$ \\
\end{tabbing}

This invariant ensures that the reserved amount is less than or equal to the total budget, as all numbers are non-negative natural numbers.
The functions \trm{bids-for-budget}, \trm{invoices-for-budget} and \trm{offers-for-budget} compute the multiset of associated bids, invoices or offers, for a particular budget.
The functions \trm{budget-total}, \trm{budget-remain}, \trm{bid-price}, and so on, are accessor functions to retrieve components of the fact values.
